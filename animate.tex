% Copyright 2007--2010 Alexander Grahn
%
% This material is subject to the LaTeX Project Public License. See
%    http://mirrors.ctan.org/help/Catalogue/licenses.lppl.html
% for the details of that license.
%
\documentclass[a4paper]{article}
\frenchspacing
\usepackage[UKenglish]{babel}
\usepackage{pst-3dplot}
\usepackage{pst-node,pst-plot,pst-tools,pst-text,pst-ode}
\usepackage{media9}
\usepackage{animate}
\usepackage{intcalc}
\usepackage{graphicx}
\graphicspath{{files/}}
\addmediapath{files}
\usepackage{color}
\usepackage{textcomp}
\usepackage{lmodern}
\usepackage{mflogo}
\usepackage[T1]{fontenc}
\usepackage[protrusion]{microtype}
\usepackage{ifpdf}
\usepackage{parskip}
\usepackage{tabls}
\usepackage{multirow}
\usepackage[bookmarksnumbered,linktocpage]{hyperref}
\usepackage{breakurl}
\usepackage{amsmath}
\usepackage{fancyvrb}
\usepackage{tabularx}
\usepackage[all]{hypcap}
\clubpenalty=10000
\widowpenalty=10000
\displaywidowpenalty=10000
\renewcommand{\textfraction}{0.0}
\renewcommand{\topfraction}{1.0}
\renewcommand{\bottomfraction}{1.0}

\makeatletter
\renewcommand{\fnum@figure}[1]{\figurename~\thefigure}
\newcommand\myparagraph{\@startsection{paragraph}{3}{\z@}%
                                     {\parskip}%
                                     {0.001\parskip}%
                                     {\itshape\normalsize}}
\makeatother

\ifpdf
  \makeatletter
  \@latex@error{This document contains PSTricks code.\MessageBreak Please compile it using the following commands:\MessageBreak\MessageBreak latex, dvips, ps2pdf -dNOSAFER\MessageBreak\MessageBreak Don't use option `-Ppdf` with dvips}
  \makeatother
\fi

\def\keywords{include portable PDF animation animated PDF animating embed animated graphics LaTeX pdfLaTeX LuaLaTeX PSTricks pgf TikZ LaTeX-picture MetaPost inline graphics vector graphics animated GIF LaTeX dvips ps2pdf dvipdfmx XeLaTeX JavaScript Adobe Reader PDF-XChange Viewer}
\hypersetup{
  pdftitle={The animate Package},
  pdfsubject={Documentation},
  pdfauthor={Alexander Grahn},
  pdfkeywords={\keywords},
  citebordercolor={0 0 1},
  filebordercolor={0 0 1},
  linkbordercolor={0 0 1},
  menubordercolor={0 0 1},
  pagebordercolor={0 0 1},
  urlbordercolor={0 0 1},
%  pdfpagelayout=SinglePage
}

\def\XeLaTeX{X\kern-.1667em\lower.5ex\hbox{\reflectbox{E}}\kern-.125em\LaTeX}
\def\pXepLaTeX{(X\kern-.1667em\lower.5ex\hbox{\reflectbox{E}})\kern-.125em\LaTeX}

\def\parsedate#1/#2/#3\relax{
  \def\year{#1}
  \def\month{#2}
  \def\day{#3}
}

\hyphenation{Ja-va-Script}
\begin{document}
\title{The {\sffamily animate} Package}
\begingroup
\makeatletter
\def\@anim@sanitize{}\def\@anim@endsanitize{}
\expandafter\parsedate\@anim@version\relax %set current date to package date
\makeatother
\author{Alexander Grahn \animategraphics[autoplay,loop,height=1.8ex]{8}{bye_}{0}{3}\protect\footnote{Animated GIF taken from \href{http://www.phpBB.com}{phpBB} forum software and burst into a set of EPS files using \href{http://www.imagemagick.org}{ImageMagick} before embedding.} \protect\footnote{Animations may run slowly if viewed in the Adobe Reader browser plugin.}\\[1ex]\includegraphics[height=2.7ex]{mailto}}
\maketitle
\endgroup
\begin{abstract}
\raggedright

\noindent A LaTeX package for creating portable, JavaScript driven PDF animations from sets of vector graphics or raster image files or from inline graphics.
\vskip 0.2\baselineskip

\emph{Keywords}: \keywords
\end{abstract}

\tableofcontents

%\newpage
\section{Introduction}
This package provides an interface to create PDFs with animated content from sets of graphics or image files, from inline graphics, such as \LaTeX-picture, PSTricks or pgf/TikZ generated pictures, or just from typeset text. Unlike standard movie/video formats, package `animate' allows for animating vector graphics. The result is roughly similar to the SWF (Flash) format, although not as space-efficient.

Package `animate' supports the usual PDF making workflows, i.\,e. pdf\LaTeX{}, Lua\LaTeX{}, \LaTeX{} $\rightarrow$ \verb+dvips+ $\rightarrow$ \verb+ps2pdf+/Distiller and \pXepLaTeX{} $\rightarrow$ \verb+(x)dvipdfmx+.

The final PDF can be viewed in current Adobe Readers on all supported platforms or in PDF-XChange Viewer.

%The `animate' package relies on Optional Content Groups (OCG), also known as PDF layers, which is a feature provided by the PDF-1.5 specification. Each frame of an animation is associated with an OCG that is hidden or made visible in a dynamic fashion by means of Adobe Reader's built-in JavaScript engine.

\section{Requirements}
\begin{trivlist}
\item $\varepsilon$-\TeX
\item pdf\TeX{}, version $\ge1.20$ or Lua\TeX{} for direct PDF output
\item Ghostscript, version $\ge8.31$ or Adobe Distiller for PS to PDF conversion
% \item Ghostscript, version $\ge9.15$ or Adobe Distiller for PS to PDF conversion
\item dvipdfmx, version $\ge20080607$ for DVI to PDF conversion
\item Adobe Reader, version $\ge7$ or PDF-XChange Viewer
\end{trivlist}

\section{Installation}
Unzip the file \href{http://mirrors.ctan.org/install/macros/latex/contrib/animate.tds.zip}{`animate.tds.zip'} into the local TDS root directory which can be found by running `\verb+kpsewhich -var-value TEXMFLOCAL+' on the command line.

After installation, update the filename database by running `\verb+texhash+' on the command line.

\TeX-Live and MiK\TeX{} users should run the package manager of their system for installation.

\section{Using the package}
First of all, read Section~\ref{sect:bugs} on problems related to this package. Then, invoke the package by putting the line
\begin{verbatim}
\usepackage[<package options>]{animate}
\end{verbatim}
to the preamble of your document, i.\,e. somewhere between \verb+\documentclass+ and \verb+\begin{document}+.

`animate' honours the package options:
\begin{verbatim}
dvipdfmx
xetex
autoplay
autopause
autoresume
loop
palindrome
draft
final
controls
buttonsize=<size>
buttonbg=<colour>
buttonfg=<colour>
step
useocg
poster[=first | none | last]
\end{verbatim}
Except for `\verb+dvipdfmx+' and `\verb+xetex+', the options above are also available (among others) as command options and will be explained shortly. However, if used as package options they have global scope, taking effect on all animations in the document. In turn, command options locally override global settings. Options without an argument are boolean options and can be negated, with the exception of package-only options `\verb+dvipdfmx+' and `\verb+xetex+', by appending `\verb+=false+'.

If PDF is generated via DVI and Postscript by the command sequence \verb+latex+ $\rightarrow$ \verb+dvips+ $\rightarrow$ \verb+ps2pdf+, the `graphicx' package is required. \begin{animateinline}[autoplay,loop]{1}\strut\emph{Important:}\newframe[3]\end{animateinline} The \verb+dvips+ option `\verb+-Ppdf+' should \emph{not} be set when converting the intermediate DVI into Postscript. If you cannot do without, put `\verb+-D 1200+' \emph{after} `\verb+-Ppdf+' on the command line. Users of \LaTeX-aware text editors with menu-driven toolchain invocation, such as \TeX{}nicCenter, should check the configuration of the \verb+dvips+ call.

%Sometimes, if the same animation sequence made from graphics files is to be embedded multiple times into the document, more compact PDF output may be obtained by passing option `\verb+-dMaxInlineImageSize=0+' to \verb+ps2pdf+.

\XeLaTeX{} and \verb+dvipdfmx+ require the `graphicx' package to be loaded. While \XeLaTeX{} will be auto-detected (package option `\verb+xetex+' is optional), `animate' and `graphicx' need the package option `\verb+dvipdfmx+' in the case of \verb+dvipdfmx+.

Occasionally, a second \LaTeX{} run may be necessary to resolve internally created object references. An appropriate warning will be issued in such a case.

\section{The user interface}
Package `animate' provides the command
\begin{verbatim}
\animategraphics[<options>]{<frame rate>}{<file basename>}{<first>}{<last>}
\end{verbatim}
and the environment
\begin{verbatim}
\begin{animateinline}[<options>]{<frame rate>}
    ... typeset material ...
\newframe[<frame rate>]
    ... typeset material ...
\newframe*[<frame rate>]
    ... typeset material ...
\newframe
\multiframe{<number of frames>}{[<variables>]}{
  ... repeated (parameterized) material ...
}
\end{animateinline}
\end{verbatim}

While \verb+\animategraphics+ can be used to assemble animations from sets of existing graphics files or from multipage PDF, the environment `\verb+animateinline+' is meant to create the animation from the typeset material it encloses. This material can be pictures drawn within the \LaTeX{} `\verb+picture+' environment or using the advanced capabilities of PSTricks or pgf/TikZ. Even ordinary textual material may be animated in this way. The parameter \verb+<frame rate>+ specifies the number of frames per second of the animation.

The \verb+\newframe+ command terminates a frame and starts the next one. It can be used only inside the `\verb+animateinline+' environment. There is a starred variant, \verb+\newframe*+. If placed after a particular frame, it causes the animation to pause at that frame. The animation continues as normal after clicking it again. Both \verb+\newframe+ variants take an optional argument that allows the frame rate to be changed in the middle of an animation.

The \verb+\multiframe+ command allows the construction of loops around pictures. The first argument \verb+<number of frames>+ does what one would expect it to do, the second argument \verb+<variables>+ is a comma-separated list of variable declarations. The list may be of arbitrary, even zero, length. Variables may be used to parameterize pictures which are defined in the loop body (third argument of \verb+\multiframe+). A single variable declaration has the form
\begin{verbatim}
  <variable name>=<initial value>+<increment>
\end{verbatim}
\verb+<variable name>+ is a sequence of one or more letters \emph{without} a leading backslash\footnote{This is different from {\tt\string\multido} (package `multido') where variable names have a leading `{\tt\textbackslash}' in the declaration.}. The first (and possibly only) letter of the variable name determines the type of the variable. There are three different types: integers (`\verb+i+', `\verb+I+'), reals (`\verb+n+', `\verb+N+', `\verb+r+', `\verb+R+') and dimensions or \LaTeX{} lengths (`\verb+d+', `\verb+D+'). Upon first execution of the loop body, the variable takes the value \verb+<initial value>+. Each further iteration increments the variable by \verb+<increment>+. Negative increments must be preceded by `\verb+-+'. Here are some examples: `\verb:i=1+2:', `\verb:Rx=10.0+-2.25:', `\verb:dim=20pt+1ex:'. Within the loop body, variables are expanded to their current value by prepending a backslash to the variable name, that is \verb+\i+, \verb+\Rx+ and \verb+\dim+ according to the previous examples. \verb+\multiframe+ must be surrounded by \verb+\begin{animateinline}+ and \verb+\end{animateinline}+ or by any of the \verb+\newframe+ variants. Two consecutive \verb+\multiframe+ commands must be separated by one of the \verb+\newframe+ variants.

By default, the animation is built frame by frame in the order of inclusion of the embedded material. However, extended control of the order of appearance, superposition and repetition of the material is available through the `\verb+timeline+' option (see Section~\ref{sect:timeline}).

\myparagraph*{Sets of graphics files}
All files of the sequence should exist and be consecutively numbered. (Exception to this rule is allowed in connection with the `\verb+every+' option, see below.) {\tt <file base\-name>} is the leftmost part of the file name that is common to all members of the sequence. \verb+<first>+ is the number of the first and \verb+<last>+ the number of the last file in the set. If \verb+<first>+ is greater than \verb+<last>+, files are embedded in reverse order. File names may be simply numbered, such as $0\dots99$. If there are leading zeros, make sure that all file numbers have the same number of digits, such as $0000\dots0099$, and that the \verb+<first>+ and \verb+<last>+ arguments are filled in accordingly.

Command \verb+\graphicspath{}+ from the `graphicx' package can be used to specify directories to be browsed for graphics files.

No file name extension may be specified as a parameter. The possible file formats depend on the output driver being used. In the case of \LaTeX{}+\verb+dvips+, files with the `eps' extension are at first searched for, followed by `mps' (\MP-generated Postscript) and `ps'. With pdf\LaTeX{} and Lua\LaTeX{} the searching order is: (1)~`pdf', (2)~`mps', (3)~`png', (4)~`jpg', (5)~`jpeg', (6)~`jbig2', (7)~`jb2', (8)~`jp2'\footnotemark[1], (9)~`j2k'\footnotemark[1], (10)~`jpx'\footnotemark[1]\footnotetext[1]{Only Lua\LaTeX{} currently supports JPEG2000.} and with \XeLaTeX{} or \LaTeX{}+\verb+dvipdfmx+: (1)~`pdf', (2)~`mps', (3)~`eps', (4)~`ps', (5)~`png', (6)~`jpg', (7)~`jpeg', (8)~`bmp'. That is, files capable of storing vector graphics are found first. Make sure that all file names have \emph{lower case} extensions.

For example, given the sequence `frame\_5.png' through `frame\_50.png' from a possibly larger set that shall be used to build an animation running at 12 frames per second, the correct inclusion command would read
\begin{verbatim}
  \animategraphics{12}{frame_}{5}{50}
\end{verbatim}

\myparagraph*{Multipage PDF {\rm(pdf\LaTeX, \XeLaTeX)} and JBIG2 {\rm(pdf\LaTeX)} inclusion}
If the file `\verb+<file basename>+.(pdf|jbig2|jb2)' exists (again, no file name extension may be specified), it is taken as a multipage document where each page represents one frame of the animation. In this case, the last two arguments, \verb+<first>+ \& \verb+<last>+, are interpreted differently from above; they specify a zero-based range of pages to be included in the animation. Either or both of them may be omitted, `\verb+{}+', in which case they default to $0$ and $n-1$, where $n$ is the total number of available pages. Arguments that fall outside this range are automatically corrected to the actual limits. If \verb+<first>+ is greater than \verb+<last>+, pages are embedded in reverse order.

For example, the line
\begin{verbatim}
  \animategraphics{12}{frames}{}{}
\end{verbatim}
would create an animation from all pages of the file `frames.pdf', running at 12 fps.

\section{Command options}
The following options to \verb+\animategraphics+ and `\verb+animateinline+' have been provided:
\subsection{Basic options}
\begin{verbatim}
label=<label text>
\end{verbatim}
The animation is given a label, \verb+<label text>+, which should be unique. Labelling an animation enables its JavaScript programming interface by defining \verb+anim['<label text>']+, which is a JavaScript reference to the animation object. The animation object provides a number of properties and methods that can be used for controlling the animation playback from within user defined JavaScript. For details, see Sect.~\ref{sect:api}.
\begin{verbatim}
poster[=first | none | last]
\end{verbatim}
Specifies which frame (first, last or none) to display and print if the animation is not activated. The first frame is shown by default. Thus `\verb+poster+' or `\verb+poster=first+' need not be explicitly set.
\begin{verbatim}
every=<num>
\end{verbatim}
Build animation from every \verb+<num>+\emph{th} frame only. Skipped frames are discarded and not embedded into the document. In the case of \verb+\animategraphics+, skipped input files may be missing.
\begin{verbatim}
autopause
\end{verbatim}
Pause animation when the page is closed, instead of stopping and rewinding it to the default frame.
\begin{verbatim}
autoplay
\end{verbatim}
Start animation after the page has opened. Also resumes playback of a previously paused animation.
\begin{verbatim}
autoresume
\end{verbatim}
Resume previously paused animation when the page is opened again.
\begin{verbatim}
loop
\end{verbatim}
The animation restarts immediately after reaching the end.
\begin{verbatim}
palindrome
\end{verbatim}
The animation continuously plays forwards and backwards.
\begin{verbatim}
step
\end{verbatim}
Step through the animation one frame at a time per mouse-click. The \verb+<frame rate>+ argument will be ignored.
\begin{verbatim}
width=<width>
height=<height>
depth=<depth>
\end{verbatim}
Resize the animation widget. Option `\verb+depth+' specifies how far the animation widget should extend below the base line of the running text. If only one or two of these options are given, the remaining, unspecified dimensions are scaled to maintain the aspect ratio. Any valid \TeX{} dimension is accepted as a parameter. In addition, the length commands \verb+\width+, \verb+\height+, \verb+\depth+ and \verb+\totalheight+ can be used to refer to the original dimensions of the animation widget which correspond to the size of the first frame of the animated sequence.
\begin{verbatim}
scale=<factor>
\end{verbatim}
Scales the animation widget by \verb+<factor>+.
\begin{verbatim}
bb=<llx> <lly> <urx> <ury>
\end{verbatim}
(\verb+\animategraphics+ only, requires package `graphicx'.) The four, space separated arguments set the bounding box of the graphics files. Units can be omitted, in which case `bp' (Postscript points) is assumed.
\begin{verbatim}
viewport=<llx> <lly> <urx> <ury>
\end{verbatim}
(\verb+\animategraphics+ only, requires package `graphicx'.) This option takes four arguments, just like `\verb+bb+'. However, in this case the values are taken relative to the origin specified by the bounding box in the graphics files.
\begin{verbatim}
trim=<left> <bottom> <right> <top>
\end{verbatim}
(\verb+\animategraphics+ only, requires package `graphicx'.) Crops graphics at the edges. The four lengths specify the amount to be removed from or, if negative values have been provided, to be added to each side of the graphics.
\begin{verbatim}
controls
\end{verbatim}
Inserts control buttons below the animation widget. The meaning of the buttons is as follows, from left to right: stop \& first frame, step backward, play backward, play forward, step forward, stop \& last frame, decrease speed, default speed, increase speed. Both `play' buttons are replaced by a large `pause' button while the animation is playing.
\begin{verbatim}
buttonsize=<size>
\end{verbatim}
Changes the control button height to \verb+<size>+, which must be a valid \TeX{} dimension. The default button height is \verb+1.44em+ and thus scales with the current font size.
\begin{verbatim}
buttonbg=<colour>
buttonfg=<colour>
\end{verbatim}
By default, control button widgets are drawn with black strokes on transparent background. The background can be turned into a solid colour by the first option, while the second option specifies the stroke colour. The parameter \verb+<colour>+ is an array of colon-(\verb+:+)-separated numbers in the range from 0.0 to 1.0. The number of array elements determines the colour model in which the colour is defined: (1) gray value, (3) RGB, (4) CMYK. For example, `\verb+1+', `\verb+1:0.5:0.2+' and `\verb+0.5:0.3:0.7:0.1+' are valid colour specifications.
\begin{verbatim}
draft
final
\end{verbatim}
With `\verb+draft+' the animation is not embedded. Instead, a box with the exact dimensions of the animation is inserted. Option `\verb+final+' does the opposite as it forces the animation to be built and embedded. Both options can be used to reduce compilation time during authoring of a document. To get the most out of them it is recommended to set `\verb+draft+' globally as a package or class option and to set `\verb+final+' locally as a command option of the animation that is currently worked on. After the document has been finished, the global `\verb+draft+' option can be removed to embed all animations.
\begin{verbatim}
useocg
\end{verbatim}
Use an alternative animation method based on Optional Content Groups (OCGs, also known as PDF Layers). May result in slower animations.
\begin{verbatim}
measure
\end{verbatim}
Measures the frame rate during one cycle of the animation. (For testing purposes.)
\begin{verbatim}
begin={<begin text>}
end={<end text>}
\end{verbatim}
(`\verb+animateinline+' only.) \verb+<begin text>+ and \verb+<end text>+ are inserted into the code at start and end of each frame. Mainly used for setting up some drawing environment, such as
\begin{verbatim}
  begin={\begin{pspicture}(... , ...)(... , ...)},
  end={\end{pspicture}}
\end{verbatim}
A short note on the `\verb+tikzpicture+' environment: Unlike `\verb+pspicture+', the `{\tt tikz\-pic\-ture}' environment is able to determine its size from the graphical objects it encloses. However, this may result in differently sized frames of a sequence, depending on the size and position of the graphical objects. Thus, in order to ensure that all frames of the sequence be displayed at the same scale in the animation widget, a common bounding box should be shared by the frames. A bounding box can be provided by means of an invisible `\verb+rectangle+' object:
\begin{verbatim}
  begin={
    \begin{tikzpicture}
    \useasboundingbox (... , ...) rectangle (... , ...);
  },
  end={\end{tikzpicture}}
\end{verbatim}

\clearpage
\subsection{The `{\tt timeline}' option}\label{sect:timeline}
\begin{verbatim}
timeline=<timeline file>
\end{verbatim}
\verb+<timeline file>+ is a plain text file whose contents determines the order of appearance of the embedded material during the animation. It allows the user to freely rearrange, repeat and overlay the material at any point of the animation. This may greatly reduce the file size of the resulting PDF, as objects that do not change between several or all frames, such as coordinate axes or labels, can be embedded once and re-used in other frames of the animation. (Technically, this is done by the XObject referencing mechanism of PDF.)

If a timeline is associated with the animation, the graphics files or inline graphics embedded by \verb+\animategraphics+ and `\verb+animateinline+' no longer represent the actual frames of the animation. Rather, they are a collection of \emph{transparencies} that can be played with at will. However, it is now up to the author's responsibility to construct a timeline that makes use of \emph{each} of those transparencies and to put them into a sensible order. In order to identify the transparencies within the timeline file, they are numbered in the order of their inclusion, starting at zero.

A timeline-based animation can be thought of as a \emph{living stack} of translucent transparencies. Each animation frame is a snapshot of the stack viewed \emph{from above}. Transparencies are usually put on top of that stack and stay there for a given number of frames before expiring (becoming invisible). The lifetime of each transparency within the stack can be set individually. Once expired, a transparency can be put on the stack again, if desired. The stack may also be divided into an arbitrary number of \emph{sub-stacks} to facilitate the creation of layers, such as background, foreground and intermediate layers. Sub-stacks allow the insertion of transparencies at depth positions of the global stack other than just the top. It is important to keep the stack-like nature of animations in mind because graphical objects on transparencies at higher stack positions overlay the content of transparencies at lower stack positions.

\myparagraph*{General structure of the timeline file}
Each line of the timeline file that is not blank and which does not begin with a comment (`\verb+%+') corresponds to \emph{one} frame of the animation. There may be more transparencies than animation frames and vice-versa. A frame specification consists of three or four colon-(\verb+:+)-separated fields:
\begin{verbatim}
  [*]:[<frame rate>]:[<transparencies>][:<JavaScript>]
\end{verbatim}
While any field may be left blank, the first two colons are mandatory. The \verb+<JavaScript>+ field is explained on p.~\pageref{sect:jsfield}.

An asterisk (`\verb+*+') in the leftmost field causes the animation to pause at that frame, very much as a \verb+\newframe*+ would do; a number in the second field changes the frame rate of the animation section that follows. In connection with the `\verb+timeline+' option, the asterisk extension and the optional \verb+<frame rate>+ argument of \verb+\newframe+ cease to make sense and will be tacitly ignored if present.

The third field \verb+<transparencies>+ is a comma-separated \emph{list} of \emph{transparency specifications} that determines the transparencies to be put on the stack. Semicolons (\verb+;+) are used to separate sub-stacks (= layers) from each other. A \emph{single} transparency specification obeys the syntax
\begin{verbatim}
  <transparency ID>[x<number of frames>]
\end{verbatim}
where \verb+<transparency ID>+ is an integer number that identifies the transparency to be drawn into the current animation frame. As pointed out above, the transparencies are consecutively numbered in the order of their inclusion, starting at zero. The optional postfix `\verb+x<number of frames>+' specifies the number of consecutive frames within which the transparency is to appear. If omitted, a postfix of `\verb+x1+' is assumed, which causes the transparency to be shown in the current frame only. Obviously, \verb+<number of frames>+ must be a non-negative integer number. The meaning of postfix `\verb+x0+' is special; it causes the transparency to be shown in all frames, starting with the current one, until the end of the animation or until the animation sub-stack to which it belongs is explicitly cleared.

The letter `\verb+c+', if put into \verb+<transparencies>+, clears an animation sub-stack, that is, it causes all transparencies added so far to be removed from the sub-stack, overriding any \verb+<number of frames>+ value. The effect of `\verb+c+' is restricted to the sub-stack in which it appears. Thus, a `\verb+c+' must be applied to every sub-stack if the complete animation stack is to be cleared. Moreover, if applied, `\verb+c+' should go into the first position of the transparency list of a sub-stack because \emph{everything} in the sub-stack up to `\verb+c+' will be cleared.

%If a frame is composed of more than one transparency, transparency specifications on the left of the input line are closer to the background and will be overprinted by those on the right of the input line or which appear on subsequent lines of the timeline file. That is, the depth \emph{de}creases from left to right within \verb+<transparencies>+ as well as in top-down direction within the timeline file.
%Also, if there are transparency specifications which span several frames (using postfix `\verb+x<number of frames>+'), they will be overprinted by transparency specifications that appear on subsequent lines in the timeline file. That is, the depth decreases in top-down direction within the timeline file.

\myparagraph*{Timeline example with a single animation stack}
Table~\ref{tab:single} is an example of a single-stack animation. It lists the contents of a timeline file together with the resulting stack of transparencies. Note how the stack is strictly built from the bottom up as transparency specifications are read from left to right and line by line from the timeline file. In frame No. 4, the stack is first cleared before new transparencies are deposited on it. Also note that the stack is viewed from above and transparencies in higher stack position overprint the lower ones.

\begin{table}[ht]\centering
\caption{Timeline example of a single-stack animation}\label{tab:single}
\begin{tabular}[t]{c|l|c}\hline
frame No. & timeline file & transparency stack\\\hline\hline
\multirow{2}{*}{0} & \multirow{2}{*}{\tt::0x0,1x2} & ---------1---------\\&&---------0---------\\\cline{1-1}\cline{3-3}
\multirow{3}{*}{1} & \multirow{3}{*}{\tt::2} & ---------2---------\\&&---------1---------\\&&---------0---------\\\cline{1-1}\cline{3-3}
\multirow{2}{*}{2} & \multirow{2}{*}{\tt::3} & ---------3---------\\&&---------0---------\\\cline{1-1}\cline{3-3}
\multirow{2}{*}{3} & \multirow{2}{*}{\tt::4} & ---------4---------\\&&---------0---------\\\cline{1-1}\cline{3-3}
\multirow{2}{*}{4} & \multirow{2}{*}{\tt::c,5x0,6} & ---------6---------\\&&---------5---------\\\cline{1-1}\cline{3-3}
\multirow{2}{*}{5} & \multirow{2}{*}{\tt::7} & ---------7---------\\&&---------5---------\\\cline{1-1}\cline{3-3}
\multirow{2}{*}{6} & \multirow{2}{*}{\tt::8} & ---------8---------\\&&---------5---------\\\cline{1-1}\cline{3-3}
\multirow{2}{*}{7} & \multirow{2}{*}{\tt::9} & ---------9---------\\&&---------5---------\\\hline
\end{tabular}
\end{table}

Figures~\ref{fig:taylor} and \ref{fig:lorenz} in Section~\ref{sect:examples} are animation examples with a single transparency stack.

\myparagraph*{Grouping objects into layers (= sub-stacks) using `{\tt;}'}
Due to the stack-like nature of the animation, the position of a transparency specification in the timeline file determines its \emph{depth} level in relation to other transparencies. The timeline file is processed line by line and from left to right. In a single-stack animation, the stack is strictly built from the bottom up, such that earlier transparencies are overprinted by more recent ones. This may turn out to be inconvenient in certain situations. For example, it might be desirable to change the background image in the middle of an animation without affecting objects that are located in the foreground. For this purpose, transparency specifications can be grouped into layers or sub-stacks using the semicolon (\verb+;+) as a separator. New transparencies can now be put on top of the individual sub-stacks. After a line of the timeline file has been processed, the global stack is built by placing the sub-stacks on top of the other. Again, the left-to-right rule applies when determining the height of the sub-stacks in relation to each other within the global stack.

The layer concept is best illustrated by an example. In the timeline of Table~\ref{tab:multi}, transparencies are grouped into two sub-stacks only. One is reserved for the background images, transparencies No. 0 \& 1, to be exchanged in frame No. 3, as well as for two other transparencies, No. 7 \& 8, to be interspersed in frame No. 1. A second sub-stack takes the foreground objects that are successively added to the scene. The dotted lines in the third column of the table just mark the border between the two sub-stacks. In frame No. 3, `\verb+c+' first clears the bottom sub-stack before the new background image is inserted. (Instead, `\verb+x3+' could have been used with transparency No. 0 in frame No. 0.) As can be seen in the specifications of frames No. 2 \& 4, sub-stacks need not be explicitly populated; the leading semicolons just ensure the proper assignment of transparencies to animation sub-stacks.

\begin{table}[hb]\centering
\caption{Timeline example with two sub-stacks}\label{tab:multi}
\begin{tabular}[t]{c|l|c}\hline
frame No. & timeline file & transparency stack\\\hline\hline
\multirow{3}{*}{0} & \multirow{3}{*}{\tt::\ \ 0x0 ; 2x0} & ---------2---------\\&&{\tiny\dotfill}\\&&---------0---------\\\cline{1-1}\cline{3-3}
\multirow{6}{*}{1} & \multirow{6}{*}{\tt::7,8x2 ; 3x0} & ---------3---------\\&&---------2---------\\&&\tiny\dotfill\\&&---------8---------\\&&---------7---------\\&&---------0---------\\\cline{1-1}\cline{3-3}
\multirow{6}{*}{2} & \multirow{6}{*}{\tt::\ \ \ \ \ \ ; 4x0} & ---------4---------\\&&---------3---------\\&&---------2---------\\&&\tiny\dotfill\\&&---------8---------\\&&---------0---------\\\cline{1-1}\cline{3-3}
\multirow{6}{*}{3} & \multirow{6}{*}{\tt::c,1x0 ; 5x0} & ---------5---------\\&&---------4---------\\&&---------3---------\\&&---------2---------\\&&\tiny\dotfill\\&&---------1---------\\\cline{1-1}\cline{3-3}
\multirow{7}{*}{4} & \multirow{7}{*}{\tt::\ \ \ \ \ \ ; 6x0} & ---------6---------\\&&---------5---------\\&&---------4---------\\&&---------3---------\\&&---------2---------\\&&\tiny\dotfill\\&&---------1---------\\\hline
\end{tabular}
\end{table}

%\small
%\begin{verbatim}
%  % <--layer 1--> <--layer 2-->
%  %
%  ::    0x49     ;   2x0,3x0    % transparency `0' used as background
%  ::             ;   4x0,5x0    % image during the first 49 frames
%  ::             ;   6x0,7x0
%  etc...
%  ::             ;  98x0,99x0
%  ::    1x0      ; 100x0,101x0  % transparency `1' used as new background
%  ::             ; 102x0,103x0  % image until end of animation
%  ::             ; 104x0,105x0
%  etc...
%\end{verbatim}
%\normalsize
%
%Note that \emph{without} setting up two layers, that is, by replacing the semicolons with commas, the foreground objects (transparencies 2 through 99) which have been added during the first 49 frames would be overprinted by the new background image, transparency 1, from frame 50 onward.

See the second animation, Fig.~\ref{fig:scarab}, in Section~\ref{sect:examples} for a working example that makes use of the timeline and the layer concept.

\myparagraph*{Associate JavaScript actions with animation frames}\label{sect:jsfield}
The optional fourth field \verb+<JavaScript>+ in a frame specification takes JavaScript code to be executed upon display of that frame. This could be used, for instance, to play a sound that was embedded using the `media9' \LaTeX{} package \cite{media9} or to execute JavaScript methods of the animation object. A non-trivial example is looping over a sub-range of frames which can be programmed by setting the `\verb+pageNum+' property of the animation object. See Sect.~\ref{sect:api} for details of the animation programming interface.

The backslash `\verb+\+' and percent `\verb+%+' characters retain their special meaning from \LaTeX{} and must be escaped by a backslash `\verb+\+' in the JavaScript code. The same applies to unbalanced braces `\verb+{+' and `\verb+}+'. Thus, a code line such as
\begin{verbatim}
  console.println('{}%}{\n');
\end{verbatim}
would have to look like
\begin{verbatim}
  console.println('{}\%\}\{\\n'); 
\end{verbatim}
in the timeline file. The first pair of braces are balancing themselves and do not need to be escaped.

Note that JavaScript is executed at the start of displaying the frame. If something is to be executed at the end of a particular frame, the \verb+<JavaScript>+ field should be added to the next frame in the timeline file. However, this is not possible for the last frame in a timeline file. Here, the `\verb+setTimeOut+' method can be used to delay the execution of commands:
\begin{verbatim}
  app.setTimeOut('anim.myanim.frameNum=5;', 0.5*anim.myanim.dt)
\end{verbatim}
In this example, the 6th frame will be displayed after half of the current frame's lifetime has elapsed.

\myparagraph*{Other things to note}
When designing the timeline, care should be taken not to include a transparency more than once into the \emph{same} animation frame. Besides the useless redundancy, this may slow down the animation speed in the Reader because the graphical objects of a multiply included transparency have to be rendered unnecessarily often at the same time. `animate' is smart enough to detect multiple inclusion and issues a warning message along with the transparency ID and the frame number if it occurs. Here is an example of a poorly designed timeline:
\small
\begin{verbatim}
  ::0
  ::1x0
  ::2
  ::3
  ::4,2
  ::5,1 % bad: transparency `1' included twice
  ::6
\end{verbatim}
\normalsize
Also, `animate' finds and warns about transparencies that have never been used in an animation timeline. This may help to avoid dead code in the final PDF.

\section{Programming interface}\label{sect:api}
The package provides a simple JavaScript programming interface which gives access to the animation objects in a PDF file. A particular animation property or method can be accessed by
\begin{verbatim}
  anim['<anim label>'].<property or method>
\end{verbatim} 
or
\begin{verbatim}
  anim.<anim label>.<property or method>
\end{verbatim}
`\verb+anim+' is an array of animation object references. Animations must be labelled using the `\verb+label=...+' command option in order to be present in the `\verb+anim+' array. As usual, properties and methods are accessed via the dot notation. Properties and methods of the animation object are summarized in Tables \ref{tab:props} and \ref{tab:methods}.

One potential use of the JavaScript interface could be within a timeline file associated with an animation. For example, loops over a sub-range of frames can be programmed by setting the `\verb+frameNum+' property. See Section~\ref{sect:timeline}, p.~\pageref{sect:jsfield} for details.

Also, the programming interface can be used to create custom buttons for playback control. The command \verb+\mediabutton+ from the `media9' package provides a convenient way for achieving this:
\begin{verbatim}
  \usepackage{media9}
  \usepackage{animate}
  ...

  \animategraphics[label=my_anim]{12}{...}{...}{...}

  \mediabutton[
    jsaction={anim['my_anim'].playFwd();}
  ]{\fbox{Play}}
  \mediabutton[
    jsaction={anim['my_anim'].frameNum=5;}
  ]{\fbox{Goto 6th frame}}
\end{verbatim}

\begin{table}[hbp]
\centering
\caption{Animation object properties}\label{tab:props}
\begin{tabularx}{\linewidth}[t]{lllX}\hline
name & type & access & description\\\hline\hline
{\tt numFrames} & Integer & read-only & Holds the total number of animation frames.\\
{\tt frameNum}  & Integer & read+write & Gets or sets the current frame being/to\,be displayed. Note that the values are zero-based. Assigning values less than zero or greater than {\tt numFrames-1} results in an error.\\
{\tt fps} & Number & read+write & Gets or sets the animation frame rate (frames per second). Assigning values less than zero results in an error.\\
{\tt dt} & Number & read-only & Holds the time span (milliseconds) for display of the current frame.\\
{\tt isPlaying} & Boolean & read-only & Holds the value {\tt true} if the animation is currently playing, {\tt false} otherwise.\\
{\tt playsFwd} & Boolean & read-only & Holds the value {\tt true} if the animation is played forward, independently of whether the animation is currently paused or not.\\
\hline
\end{tabularx}
\end{table}

\begin{table}[hbp]
\centering
\caption{Animation object methods}\label{tab:methods}
\begin{tabularx}{\linewidth}[t]{lX}\hline
name & description\\\hline\hline
{\tt playFwd()} & Starts playing the animation in the forward direction, setting the properties {\tt isPlaying} and {\tt playsFwd} to {\tt true}.\\
{\tt playBwd()} & Starts playing the animation in the backward direction, setting {\tt isPlaying} to  {\tt true} and {\tt playsFwd} to {\tt false}.\\
{\tt pause()} & Pauses animation, setting {\tt isPlaying} to {\tt false}.\\
{\tt stopFirst()} & Stops animation and rewinds to the first frame. {\tt isPlaying} is set to false.\\
{\tt stopLast()} & Stops animation and goes to the last frame. {\tt isPlaying} is set to false.\\
\hline
\end{tabularx}
\end{table}

\clearpage
\section{Examples}
\subsection[Animations from sets of files, using `animategraphics' command]{Animations from sets of files, using {\tt \string\animategraphics} command}\label{sect:examples}
Animations in this section are made from graphics files that were prepared with \MP. Run `\verb+mpost --tex=latex+' on the files ending in `.mp' in the `files' directory to generate the graphics files. Both examples make use of the `\verb+timeline+' option to reduce the resulting PDF file size.

The first example, Fig.~\ref{fig:taylor}, originally written by Jan Hole\v{c}ek~\cite{hol}, shows the exponential function $y=e^x$ and its approximation by Taylor polynomials of different degree. Below the animation, a custom button was inserted using the JavaScript programming interface and the \verb+\mediabutton+ command from the `media9' package.

\small
\begin{verbatim}
\documentclass{article}
\usepackage{animate}
\usepackage{media9}
\usepackage{graphicx}

\begin{filecontents}{timeline.txt}
::0x0 % coordinate system & y=e^x, repeated until last frame
::1   % one blue curve per frame
::2
::3
::4
::5
::6
::7
::8
\end{filecontents}

\begin{document}

\begin{center}
  \animategraphics[
    label=taylor,
    controls, loop,
    timeline=timeline.txt
  ]{4}{exp_}{0}{8}

  \mediabutton[
    jsaction={
      if(anim['taylor'].isPlaying)
        anim['taylor'].pause();
      else  
        anim['taylor'].playFwd();
    }
  ]{\fbox{Play/Pause}}
\end{center}

\end{document}
\end{verbatim}
\normalsize
\begin{figure}\capstart
\centering
\begin{VerbatimOut}{timeline.txt}
::0x0 % coordinate system & y=e^x, repeated until last frame
::1   % one blue curve per frame
::2
::3
::4
::5
::6
::7
::8
\end{VerbatimOut}
\animategraphics[label=taylor,controls,loop,timeline=timeline.txt]{4}{exp_}{0}{8}

\mediabutton[
  jsaction={
    if(anim['taylor'].isPlaying)
      anim['taylor'].pause();
    else  
      anim['taylor'].playFwd();
  }
]{\fbox{Play/Pause}}
\caption{}\label{fig:taylor}
\end{figure}

The second, somewhat more complex example, Fig.~\ref{fig:scarab}, animates the geometric construction of a scarabaeus. In addition to the use of a timeline, it introduces the layer concept. This example is adapted from Maxime Chupin's original \MP{} source file \cite{chupin}. The present version separates stationary from moving parts of the drawing and saves them into different files. A total of 254 files, scarab\_0.mps through scarab\_253.mps, is written out by running `\verb+mpost --tex=latex+' on the source file `scarab.mp'. Files 0 through 100 contain the red line segments that make up the growing scarabaeus. Files 101 through 201 contain the moving construction lines and files 202 through 252 contain the gray lines which represent intermediate stages of the construction. The last file, No. 253, contains the coordinate axes, two stationary construction lines and the labels which do not move. A timeline file `scarab.tln' is written out on-the-fly during the \LaTeX{} run. It arranges the animation into three layers, forcing the gray lines into the background, the coordinate axes into the intermediate layer and the scarabaeus along with the moving construction lines into the foreground. The final animation consists of 101 individual frames.
\begin{figure}[t]
\centering
\newcounter{scarab}
\setcounter{scarab}{0}
\newcounter{blueline}
\setcounter{blueline}{101}
\newcounter{grayline}
\setcounter{grayline}{202}
%
\newwrite\TimeLineFile
\immediate\openout\TimeLineFile=scarab.tln
\whiledo{\thescarab<101}{
  \ifthenelse{\intcalcMod{\thescarab}{2}=0}{
    \immediate\write\TimeLineFile{::\thegrayline x0;253;\thescarab x0,\theblueline}
    \stepcounter{grayline}
  }{
    \immediate\write\TimeLineFile{::;253;\thescarab x0,\theblueline}
  }
  \stepcounter{scarab}
  \stepcounter{blueline}
}
\immediate\closeout\TimeLineFile
%
\animategraphics[
  width=0.8\linewidth,
  controls,
  loop,
  timeline=scarab.tln
]{12}{scarab_}{0}{253}
%
\caption{}\label{fig:scarab}
\end{figure}

\small
\begin{verbatim}
\documentclass{article}
\usepackage{intcalc} %defines \intcalcMod for Modulo computation
\usepackage{animate}
\usepackage{graphicx}

\newcounter{scarab}
\setcounter{scarab}{0}
\newcounter{blueline}
\setcounter{blueline}{101}
\newcounter{grayline}
\setcounter{grayline}{202}

%write timeline file
\newwrite\TimeLineFile
\immediate\openout\TimeLineFile=scarab.tln
\whiledo{\thescarab<101}{
  \ifthenelse{\intcalcMod{\thescarab}{2}=0}{
    %a gray line is added to every 2nd frame
    \immediate\write\TimeLineFile{%
      ::\thegrayline x0;253;\thescarab x0,\theblueline}
    \stepcounter{grayline}
  }{
    \immediate\write\TimeLineFile{%
      ::;253;\thescarab x0,\theblueline}
  }
  \stepcounter{scarab}
  \stepcounter{blueline}
}
\immediate\closeout\TimeLineFile

\begin{document}

\begin{center}
  \animategraphics[
    width=0.8\linewidth,
    controls, loop,
    timeline=scarab.tln
  ]{12}{scarab_}{0}{253}
\end{center}

\end{document}
\end{verbatim}
\normalsize
\subsection[Animating PSTricks graphics, using `animateinline' environment]{Animating PSTricks graphics, using `{\tt animateinline}' environment}
Fig.~\ref{fig:torus} is an inline graphics example adapted from \cite{gilg05}.
\small
\begin{verbatim}
\documentclass{article}
\usepackage{pst-3dplot}
\usepackage{animate}

%draws a torus sector
\newcommand{\torus}[2]{% #1: angle of the torus sector,
  %                      #2: linewidth of leading circle
  \psset{Beta=20,Alpha=50,linewidth=0.1pt,origin={0,0,0},unit=0.35}%
  \begin{pspicture}(-12.3,-6.3)(12.3,7)%
    \parametricplotThreeD[xPlotpoints=100](80,#1)(0,360){%
      t cos 2 mul 4 u sin 2 mul add mul
      t sin 2 mul 4 u sin 2 mul add mul
      u cos 4 mul
    }%
    \parametricplotThreeD[yPlotpoints=75](0,360)(80,#1){%
      u cos 2 mul 4 t sin 2 mul add mul
      u sin 2 mul 4 t sin 2 mul add mul
      t cos 4 mul
    }%
    \parametricplotThreeD[yPlotpoints=1,linewidth=#2](0,360)(#1,#1){%
      u cos 2 mul 4 t sin 2 mul add mul
      u sin 2 mul 4 t sin 2 mul add mul
      t cos 4 mul
    }%
  \end{pspicture}%
}

\begin{document}

\begin{center}
\begin{animateinline}[poster=last, controls, palindrome]{12}%
  \multiframe{29}{iAngle=80+10, dLineWidth=2.9pt+-0.1pt}{%
    %iAngle = 80, 90, ..., 360 degrees
    %dLineWidth = 2.9pt, 2.8pt, ..., 0.1pt
    \torus{\iAngle}{\dLineWidth}%
  }%
\end{animateinline}%
\end{center}

\end{document}
\end{verbatim}
\normalsize

\begin{figure}
\centering
\newcommand{\torus}[2]{%
  \psset{Beta=20,Alpha=50,linewidth=0.1pt,origin={0,0,0},unit=0.35}%
  \begin{pspicture}(-12.3,-6.3)(12.3,7)%
    \parametricplotThreeD[xPlotpoints=100](80,#1)(0,360){%
      t cos 2 mul 4 u sin 2 mul add mul
      t sin 2 mul 4 u sin 2 mul add mul
      u cos 4 mul
    }%
    \parametricplotThreeD[yPlotpoints=75](0,360)(80,#1){%
      u cos 2 mul 4 t sin 2 mul add mul
      u sin 2 mul 4 t sin 2 mul add mul
      t cos 4 mul
    }%
    \parametricplotThreeD[yPlotpoints=1,linewidth=#2](0,360)(#1,#1){%
      u cos 2 mul 4 t sin 2 mul add mul
      u sin 2 mul 4 t sin 2 mul add mul
      t cos 4 mul
    }%
  \end{pspicture}%
}

\begin{animateinline}[poster=last,controls,palindrome]{12}%
  \multiframe{29}{iAngle=80+10, dLineWidth=2.9pt+-0.1pt}{%
    \torus{\iAngle}{\dLineWidth}%
  }%
\end{animateinline}%
\caption{}\label{fig:torus}
\end{figure}

Another inline example, Fig.~\ref{fig:lorenz}, is an animation of the Lorenz Attractor. The Lorenz Attractor is a three-dimensional parametric curve whose coordinates are obtained by integrating the set of three ordinary differential equations
\begin{align*}
  \frac{\mathrm{d}x}{\mathrm{d}t}& = \alpha (y-x)\\
  \frac{\mathrm{d}y}{\mathrm{d}t}& = x(\beta-z)-y\\
  \frac{\mathrm{d}z}{\mathrm{d}t}& = x y - \gamma z
\end{align*}
with respect to the independent parameter $t$. The shape of the attractor strongly depends on the values chosen for the coefficients $\alpha$, $\beta$ and $\gamma$ as well as on the initial conditions, that is, the coordinates $x_0$, $y_0$ and $z_0$ of the starting point of the curve. Here we use the values $\alpha=10$, $\beta=28$, $\gamma=8/3$ and the starting point $\mathbf{x}_0=(10,10,30)$.

The right hand sides of the equations above are defined in the macro \verb+\lorenz+ as algebraic expressions. The initial value problem is solved by the macro \verb+\pstODEsolve+ from the PSTricks package `pst-ode' and plotted by the macro \verb+\parametricplotThreeD+ from the PSTricks package `pst-3dplot'.

A timeline file, written on-the-fly, is used to assemble the curve segments frame by frame to the growing attractor which, in turn, is put on top of the $x$-$y$-$z$ coordinate system. After the attractor has been completed, the transparency stack is cleared. Then, transparencies containing the complete curve and the coordinate system seen from different viewpoints are put in a row to produce the animated fly-around.

\begin{figure}[t]
\centering
%
%Lorenz' set of differential equations
\def\lorenz{%
  10*(x[1]-x[0])        | %dx/dt
  x[0]*(28-x[2]) - x[1] | %dy/dt
  x[0]*x[1] - 8/3*x[2]    %dz/dt
}%
%
%write timeline file
\newwrite\OutFile%
\immediate\openout\OutFile=lorenz.tln%
\multido{\iLorenz=0+1}{101}{%
  \immediate\write\OutFile{::\iLorenz x0}%
}%
\immediate\write\OutFile{::c,101}%
\multido{\iLorenz=102+1}{89}{%
  \immediate\write\OutFile{::\iLorenz}%
}%
\immediate\closeout\OutFile%
%
\psset{unit=0.155,linewidth=0.5pt}%
\begin{animateinline}[
  timeline=lorenz.tln,
  controls,poster=last,
  begin={\begin{pspicture}(-39,-13)(39,60)},
  end={\end{pspicture}}
]{10}
  %coordinate axes
  \psset{Alpha=120,Beta=20}%
  \pstThreeDCoor[xMax=33,yMax=33,zMax=55,linecolor=black]%
\newframe
  \pstODEsetOrRestoreState{10 10 30} %initial condition
  \multiframe{100}{i=0+1,rtZero=0+0.25,rtOne=0.25+0.25}{%
    \pstODEsaveState{savedState}%
    %compute attractor segments, stored in `lorenzXYZseg'
    \pstODEsolve[algebraic]{lorenzXYZseg}{0 1 2}{\rtZero}{\rtOne}{26}{savedState}%
	                                                                      {\lorenz}%
    %compute the whole attractor, stored in `lorenzXYZall', with the same stepping
    \ifnum\i=0\relax%
      \pstODEsolve[algebraic]{lorenzXYZall}{0 1 2}{\rtZero}{\rtOne}{26}{savedState}%
	                                                                        {\lorenz}%
    \else%
      \pstODEsolve[algebraic,append]{lorenzXYZall}{0 1 2}{\rtZero}{\rtOne}{26}%
	                                                       {savedState}{\lorenz}%
    \fi%
    \listplotThreeD[plotstyle=line]{lorenzXYZseg}%
  }%
\newframe% required between two \multiframe
  \multiframe{90}{rAlpha=116+-4}{% fly-around
    \psset{Alpha=\rAlpha,Beta=20}%
    \pstThreeDCoor[xMax=33,yMax=33,zMax=55,linecolor=black]%
    \listplotThreeD[plotstyle=line]{lorenzXYZall}%
  }%
\end{animateinline}
\caption{}\label{fig:lorenz}
\end{figure}

\small
\begin{verbatim}
\documentclass{article}
\usepackage{multido}
\usepackage{pst-3dplot}
\usepackage{pst-ode}
\usepackage{animate}

\begin{document}

%Lorenz' set of differential equations
\def\lorenz{%
  10*(x[1]-x[0])        | %dx/dt
  x[0]*(28-x[2]) - x[1] | %dy/dt
  x[0]*x[1] - 8/3*x[2]    %dz/dt
}%
%
%write timeline file
\newwrite\OutFile%
\immediate\openout\OutFile=lorenz.tln%
\multido{\iLorenz=0+1}{101}{%
  \immediate\write\OutFile{::\iLorenz x0}%
}%
\immediate\write\OutFile{::c,101}%
\multido{\iLorenz=102+1}{89}{%
  \immediate\write\OutFile{::\iLorenz}%
}%
\immediate\closeout\OutFile%
%
\psset{unit=0.155,linewidth=0.5pt}%
\begin{animateinline}[
  timeline=lorenz.tln,
  controls,poster=last,
  begin={\begin{pspicture}(-39,-13)(39,60)},
  end={\end{pspicture}}
]{10}
  %coordinate axes
  \psset{Alpha=120,Beta=20}%
  \pstThreeDCoor[xMax=33,yMax=33,zMax=55,linecolor=black]%
\newframe
  \pstODEsetOrRestoreState{10 10 30} %initial condition
  \multiframe{100}{i=0+1,rtZero=0+0.25,rtOne=0.25+0.25}{%
    \pstODEsaveState{savedState}%
    %compute attractor segments, stored in `lorenzXYZseg'
    \pstODEsolve[algebraic]{lorenzXYZseg}{0 1 2}{\rtZero}{\rtOne}{26}{savedState}%
	                                                                      {\lorenz}%
    %compute the whole attractor, stored in `lorenzXYZall', with the same stepping
    \ifnum\i=0\relax%
      \pstODEsolve[algebraic]{lorenzXYZall}{0 1 2}{\rtZero}{\rtOne}{26}{savedState}%
	                                                                        {\lorenz}%
    \else%
      \pstODEsolve[algebraic,append]{lorenzXYZall}{0 1 2}{\rtZero}{\rtOne}{26}%
	                                                       {savedState}{\lorenz}%
    \fi%
    \listplotThreeD[plotstyle=line]{lorenzXYZseg}%
  }%
\newframe% required between two \multiframe
  \multiframe{90}{rAlpha=116+-4}{% fly-around
    \psset{Alpha=\rAlpha,Beta=20}%
    \pstThreeDCoor[xMax=33,yMax=33,zMax=55,linecolor=black]%
    \listplotThreeD[plotstyle=line]{lorenzXYZall}%
  }%
\end{animateinline}

\end{document}
\end{verbatim}
\normalsize

The last inline example in Fig.~\ref{fig:metronome} is a ticking metronome written by Manuel Luque~\cite{luque12}. The short clicking sound was embedded by means of the `media9' package. Whenever the pendulum reaches one of its reversal points, playback of the sound file is started using JavaScript. The JavaScript code was inserted at the corresponding frame specifications in a timeline file. Since the PSTricks macros for drawing the metronome body and the pendulum are quite long they have been moved into an external file, \verb+files/pstmetronome.tex+. Note that the sound can be heard only on Win and Mac platforms. Even then, mileage may vary. A dual core CPU may be necessary for fluent playback.
\begin{figure}[hb]
\centering
\begin{VerbatimOut}{files/pstmetronome.tex}
%%%%%%%%%%%%%%%%%%%%%%%%%%%%%%%%%%%%%%%%%%%%%%%%%%%%%%%%%%%%%%%%%%%%%%%
% animated metronome
% this code is based to 99.9 percent on the work by Manuel Luque
% (pstricks.blogspot.com)
%%%%%%%%%%%%%%%%%%%%%%%%%%%%%%%%%%%%%%%%%%%%%%%%%%%%%%%%%%%%%%%%%%%%%%%
\makeatletter
\pst@addfams{pst-metronome}
\define@key[psset]{pst-metronome}{theta0}{\def\psk@oscmetronomethetai{#1 }}
\psset[pst-metronome]{theta0=45} % position initiale du metronome
\define@key[psset]{pst-metronome}{M}{\def\psk@oscmetronometM{#1 }}
\psset[pst-metronome]{M=25} % masse du disque en g
\define@key[psset]{pst-metronome}{m}{\def\psk@oscmetronometm{#1 }}
\psset[pst-metronome]{m=6} % masse du curseur en g
\define@key[psset]{pst-metronome}{r}{\def\psk@oscmetronomer{#1 }}
\psset[pst-metronome]{r=1} % rayon du disque en cm
\define@key[psset]{pst-metronome}{x}{\def\psk@oscmetronomex{#1 }}
\psset[pst-metronome]{x=8.4} % position du curseur en cm par rapport � l'axe
\define@key[psset]{pst-metronome}{d}{\def\psk@oscmetronomed{#1 }}
\psset[pst-metronome]{d=3.2} % distance de l'axe au centre du disque en cm
\define@key[psset]{pst-metronome}{dt}{\def\psk@oscmetronomedt{#1 }}
\psset[pst-metronome]{dt=0.01} % pas pour RK4
\define@key[psset]{pst-metronome}{nT}{\def\psk@oscmetronomenT{#1 }}
\psset[pst-metronome]{nT=1} % nombre de p�riodes repr�sent�es
%---- calculer theta(t) et thetapoint(t) --------
\def\psmetronome{\pst@object{psmetronome}}
\def\psmetronome@i{%
\begingroup%
\use@par%
  \begin@SpecialObj%
  \pstVerb{%
 /deg2rad {180 div 3.14159 mul} def
 /rad2deg {180 mul 3.14159 div} def
 /gp 9.8 def % pesanteur
 /radius \psk@oscmetronomer 1e-2 mul def % en m
 /OA \psk@oscmetronomed 1e-2 mul def % distance de l'axe au centre du disque en m
 /xC \psk@oscmetronomex 1e-2 mul def % position du curseur en m par rapport � l'axe
 /theta0 \psk@oscmetronomethetai def % en degr�s
 /theta0rad theta0 deg2rad def % en radians
 /Md \psk@oscmetronometM 1e-3 mul def % en kg
 /mc \psk@oscmetronometm 1e-3 mul def % en kg
 /dt \psk@oscmetronomedt def
 /nT \psk@oscmetronomenT def
 % moment d'inertie du m�tronome
 % J=1/2M*R^2+M*a^2+m*x^2
 /Ji {0.5 Md mul radius dup mul mul Md OA dup mul mul add mc xC dup mul mul add} def
 /AT {4
      Ji
      gp Md OA mul mc xC mul sub mul
      div
      sqrt
      mul} def
 % Pour le calcul de la p�riode
 % coefficients de l'approximation polyn�miale du calcul
 % de l'int�grale elliptique
% coefficient pour le calcul de l'int�grale elliptique
        /m0 theta0 2 div sin def
        /m1 {1 m0 dup mul sub} def
        /m2 {m1 dup mul} def
        /m3 {m2 m1 mul} def
        /m4 {m2 dup mul} def
        /m_1 {1 m1 div} def
     /EllipticK {
        0.5
        0.12498593597 m1 mul add
        0.06880248576 m2 mul add
        0.03328355376 m3 mul add
        0.00441787012 m4 mul add
        m_1 ln mul
        1.38629436112 add
        0.09666344259 m1 mul add
        0.03590092383 m2 mul add
        0.03742563713 m3 mul add
        0.01451196212 m4 mul add
      } def
/Tm {AT EllipticK mul} def
% tableau des valeurs de theta(t)
    /W 0 def % vitesse angulaire
    /theta theta0 def
    /oscillateur {sin gp Md OA mul mc xC mul sub mul neg mul Ji div} def
    /j1 {W dt mul} def
    /k1 {theta oscillateur dt mul} def
    /j2 {W k1 2 div add dt mul} def
    /k2 {theta j1 2 div rad2deg add oscillateur dt mul} def
    /j3 {W k2 2 div add dt mul} def
    /k3 {theta j2 2 div rad2deg add oscillateur dt mul} def
    /j4 {W k3 add dt mul} def
    /k4 {theta j3 rad2deg add oscillateur dt mul} def
    /theta2 {theta j1 rad2deg 2 j2 rad2deg j3 rad2deg add mul add j4 rad2deg add 6 div add} def
/tabTheta [ % pour l'animation
    0 theta0 % date angle
dt dt Tm nT mul{ %
    theta2 %
    /W2 W k1 2 k2 k3 add mul add k4 add 6 div add def
    /theta theta2 def
    /W W2 def
    }  for
        ] def
/Nvaleurs tabTheta length 2 div cvi def
    /W 0 def % vitesse angulaire
    /theta theta0 def
/tabThetaGraph [ % pour le graphique theta(t)
    0 theta0  % date angle
0 dt Tm nT mul { % pop
    theta2 % 180 div 3.14159 mul
    /W2 W k1 2 k2 k3 add mul add k4 add 6 div add def
    /theta theta2 def
    /W W2 def
    }  for
        ] def
    /W 0 def % vitesse angulaire
    /theta theta0 def
/tabThetaPoint [ % pour le graphique thetapoint(t)
    0 0  % date angle
dt dt Tm nT mul { % pop
%    theta2 % 180 div 3.14159 mul
    /W2 W k1 2 k2 k3 add mul add k4 add 6 div add def
    W2
    /theta theta2 def
    /W W2 def
    }  for
        ] def
/tabXOSC [ % oscillations par min en fonction de x
0.5 0.1 12 {/xc exch def
 /xC xc 1e-2 mul def
 xc 60 Tm div % cvi
 } for
        ] def
/tabXbattements [ % battements par min en fonction de x
3 0.1 12 {/xc exch def
 /xC xc 1e-2 mul def
 xc 60 Tm div 2 mul % cvi
 } for
        ] def
% graduation T --> x
/tabXT [ % [T,x]
 40 1 220 {/batt exch def % battements
 /Tmetronome2 120 batt div dup mul def
 /A1 16 mc mul EllipticK dup mul mul def
 /B1 gp Tmetronome2 mul mc mul def
 /C1 gp Md mul OA mul Tmetronome2 mul neg
     8 Md mul radius dup mul mul 16 Md mul OA dup mul mul add EllipticK dup mul mul add def
 /Delta B1 dup mul 4 A1 mul C1 mul sub sqrt def
 /xC1 B1 neg Delta sub 2 div A1 div def
 /xC2 B1 neg Delta add 2 div A1 div def
 xC2 0 ge {/posC xC2 def}{/posC xC1 def} ifelse
 batt posC 1e2 mul
 } for
        ] def
/xT { % pour une valeur particuli�re battement -> position du curseur
    /batt exch def
    /Tmetronome2 120 batt div dup mul def
    /A1 16 mc mul EllipticK dup mul mul def
    /B1 gp Tmetronome2 mul mc mul def
    /C1 gp Md mul OA mul Tmetronome2 mul neg
     8 Md mul radius dup mul mul 16 Md mul OA dup mul mul add EllipticK dup mul mul add def
    /Delta B1 dup mul 4 A1 mul C1 mul sub sqrt def
    /xC1 B1 neg Delta sub 2 div A1 div def
    /xC2 B1 neg Delta add 2 div A1 div def
     xC2 0 ge {/posC xC2 def}{/posC xC1 def} ifelse
     posC 1e2 mul
 } def
/xC \psk@oscmetronomex 1e-2 mul def % position du curseur en m par rapport � l'axe
/Tm {AT EllipticK mul} def
  }%
  \end@SpecialObj%
\endgroup}
%
\def\psmetronomeA{\pst@object{psmetronomeA}}
\def\psmetronomeA@i{%
\begingroup%
\use@par%
  \begin@SpecialObj%
  \pstVerb{%
 /radius \psk@oscmetronomer 1e-2 mul def % en m
 /OA \psk@oscmetronomed 1e-2 mul def % distance de l'axe au centre du disque en m
 /xC \psk@oscmetronomex 1e-2 mul def % position du curseur en m par rapport � l'axe
  }%
\psframe[fillstyle=solid](! -0.075 \psk@oscmetronomed neg)(0.075,13)
\pscircle[fillstyle=solid,fillcolor={[rgb]{0.75 0.75 0.75}}](! 0 \psk@oscmetronomed neg){!radius 1e2 mul}
\pscircle[fillstyle=solid,linewidth=0.05](0,0){0.15}
\pscircle*[linecolor=red](0,0){0.05}
% curseur
\pspolygon[fillstyle=solid](! -0.25 \psk@oscmetronomex 0.5 sub)(! -0.5 \psk@oscmetronomex 0.5 add)(!-0.075 \psk@oscmetronomex 0.5 add)(!-0.075 \psk@oscmetronomex 0.5 sub)
\pspolygon[fillstyle=solid](! 0.25 \psk@oscmetronomex 0.5 sub)(! 0.5 \psk@oscmetronomex 0.5 add)(!0.075 \psk@oscmetronomex 0.5 add)(!0.075 \psk@oscmetronomex 0.5 sub)
\pspolygon[fillstyle=solid,fillcolor=gray](! -0.25 \psk@oscmetronomex 0.5 sub)(! -0.3 \psk@oscmetronomex 0.3 sub)(! -0.075 \psk@oscmetronomex 0.3 sub)(!-0.075 \psk@oscmetronomex 0.3 add)(!0.075 \psk@oscmetronomex 0.3 add)(!0.075 \psk@oscmetronomex 0.3 sub)(!0.3 \psk@oscmetronomex 0.3 sub)(!0.25 \psk@oscmetronomex 0.5 sub)
\pscircle[fillstyle=solid](!-0.125 \psk@oscmetronomex 0.4 sub){0.08}
\pscircle[fillstyle=solid](!0.125 \psk@oscmetronomex 0.4 sub){0.08}
% fin curseur
{\psset{linecolor=red}
\psline(!-.1 \psk@oscmetronomex)(!0.1 \psk@oscmetronomex)\psline(!0 \psk@oscmetronomex 0.1 sub)(!0 \psk@oscmetronomex 0.1 add)
\psline(! -.1 \psk@oscmetronomed neg)(!0.1 \psk@oscmetronomed neg)\psline(! 0 \psk@oscmetronomed neg 0.1 sub)(!0 \psk@oscmetronomed neg 0.1 add)}
\pnode(!0 \psk@oscmetronomex){C}% curseur
\pnode(! 0 \psk@oscmetronomed neg){D}% disque
\pstextpath[c](0,-2ex){\psarcn[linestyle=none](D){1}{180}{0}}{\small\textsf{\textbf{m e t r o n o m e}}}
\pstextpath[c](0,1ex){\psarc[linestyle=none](D){1}{180}{0}}{\small\textsf{\textbf{P S t r i c k s}}}
  \end@SpecialObj%
\endgroup}
\psmetronome%
\pstVerb{/tabTempos [40 42 44 46 48 50 52 54 46 58 60 63 66 69 72 76 80 84 88 92 96 100 104 108 112 116 120 126 132 138 144 152 160 168 176 184 192 200 208] def}%

\def\metronomebody{%
  \pspolygon[fillstyle=solid,linewidth=2\pslinewidth,linearc=0.5,fillcolor=yellow!30](-5,-4.5)(5,-4.5)(1,14)(-1,14)
  \psline(1.2,4.5)(1.2,12.5)
  \psline(-1.2,4.5)(-1.2,12.5)
  \multido{\i=0+2}{20}{%
     \pstVerb{/BATT tabTempos \i\space get def}
     \psline[linecolor=red](!1.2 BATT xT)(!0.7 BATT xT)
     \uput[r](!0.7 BATT xT){\psPrintValue[PSfont=Helvetica,fontscale=6]{BATT}}
  }%
  \multido{\i=1+2}{19}{%
     \pstVerb{/BATT tabTempos \i\space get def}
     \psline[linecolor=red](!-1.2 BATT xT)(!-0.7 BATT xT)
     \uput[r](!-1.3 BATT xT){\psPrintValue[PSfont=Helvetica,fontscale=6]{BATT}}}%
  \rput(!0 40 xT){\textsf{\tiny GRAVE}}%
  \rput(!0 46 xT){\textsf{\tiny LARGO}}%
  \rput(!0 52 xT){\textsf{\tiny LENTO}}%
  \rput(!0 58 xT){\textsf{\tiny ADAGIO}}%
  \rput(!0 60 xT){\textsf{\tiny LARGETTO}}%
  \rput(!0 66 xT){\textsf{\tiny ANDANTE}}%
  \rput(!0 76 xT){\textsf{\tiny ANDANTINO}}%
  \rput(!0 84 xT){\textsf{\tiny MODERATO}}%
  \rput(!0 108 xT){\textsf{\tiny ALLEGRETTO}}%
  \rput(!0 132 xT){\textsf{\tiny ALLEGRO}}%
  \rput(!0 160 xT){\textsf{\tiny VIVACE}}%
  \rput(!0 184 xT){\textsf{\tiny PRESTO}}%
  \rput(!0 200 xT){\textsf{\tiny PRESTISSIMO}}%
}

\def\pendulum#1{%
  \pstVerb{/iA #1\space def /date tabTheta iA get def /Theta tabTheta iA 1 add get def}%
  \rput{!Theta}{\psmetronomeA}%
}
\makeatother
\end{VerbatimOut}
\begin{VerbatimOut}{metro.txt}
::0x0,1 : annotRM['click'].callAS('play');
::2
::3
::4
::5
::6
::7
::8
::9
::10
::11
::12
::13
::14
::15
::16
::17
::18
::19
::20
::21
::22
::23
::24
::25
::26 : annotRM['click'].callAS('play');
\end{VerbatimOut}
%loading metronome macros from external file
\input{files/pstmetronome}
%sound inclusion: click.mp3
\makebox[0pt][r]{\includemedia[
  width=1ex,height=1ex,
  label=click,
  addresource=click.mp3,
  activate=pageopen,transparent,noplaybutton,
  flashvars={source=click.mp3&hideBar=true}
]{}{APlayer.swf}}%
%animated metronome
\begin{animateinline}[
  controls,
  width=0.7\linewidth,
  palindrome,
  begin={\begin{pspicture}(-9.5,-5)(9.5,15)},
  end={\end{pspicture}},
  timeline=metro.txt
]{25}
  %metronome without pendulum
  \metronomebody
\newframe
  %half period of pendulum swing (26 frames)
  \multiframe{26}{i=0+4}{
    \pendulum{\i}
  }
\end{animateinline}
\caption{}\label{fig:metronome}
\end{figure}
\small
\begin{verbatim}
\documentclass[12pt]{article}
\usepackage{pstricks,pst-node,pst-plot,pst-tools,pst-text}
\usepackage{animate}
\usepackage{media9}

%writing timeline to external file
\begin{filecontents}{metro.txt}
::0x0,1 : annotRM['click'].callAS('play');
::2
::3
::4
::5
::6
::7
::8
::9
::10
::11
::12
::13
::14
::15
::16
::17
::18
::19
::20
::21
::22
::23
::24
::25
::26 : annotRM['click'].callAS('play');
\end{filecontents}

\begin{document}

\begin{center}
  %loading metronome macros from external file
  \input{files/pstmetronome}
  %
  %sound inclusion: click.mp3
  \makebox[0pt][r]{\includemedia[
    width=1ex,height=1ex,
    label=click,
    addresource=click.mp3,
    activate=pageopen,transparent,noplaybutton,
    flashvars={source=click.mp3&hideBar=true}
  ]{}{APlayer.swf}}%
  %
  %animated metronome
  \begin{animateinline}[
    controls,
    width=0.7\linewidth,
    palindrome,
    begin={\begin{pspicture}(-9.5,-5)(9.5,15)},
    end={\end{pspicture}},
    timeline=metro.txt
  ]{25}
    %metronome without pendulum
    \metronomebody
  \newframe
    %half period of pendulum swing (26 frames)
    \multiframe{26}{i=0+4}{
      \pendulum{\i}
    }
  \end{animateinline}
\end{center}

\end{document}
\end{verbatim}

\section{Bugs}\label{sect:bugs}
\begin{itemize}
  \item The maximum frame rate that can actually be achieved largely depends on the complexity of the graphics and on the available hardware. Starting with version 8, Adobe Reader appears to be somewhat slower. However, you might want to experiment with the graphical hardware acceleration feature that was introduced in Reader 8. Go to menu `Edit' $\rightarrow$ `Preferences' $\rightarrow$ `Page Display' $\rightarrow$ `Rendering' to see whether hardware acceleration is available. A 2D GPU acceleration check box will be visible if a supported video card has been detected.

  \item Animations may run very slowly if viewed in the Adobe Reader web-browser plugin. Instead, open the PDF locally in the Reader application for best results.

  \item The Adobe Reader setting `Use page cache' (menu `Edit' $\rightarrow$ `Preferences' $\rightarrow$ `Startup') should be \emph{dis}abled for version 7, while remaining \emph{en}abled beginning with version 8 (menu `Edit' $\rightarrow$ `Preferences' $\rightarrow$ `Page Display' $\rightarrow$ `Rendering').

  \item The \verb+dvips+ option `\verb+-Ppdf+' should be avoided entirely or followed by something like `\verb+-D 1200+' on the command line in order to set a sensible DVI resolution. This does \emph{not} degrade the output quality! The configuration file `config.pdf' loaded by option `\verb+-Ppdf+' specifies an excessively high DVI resolution that will be passed on to the final PDF. Eventually, Adobe Reader gets confused and will not display the frames within the animation widget.

%  \item Animations do not work if the PDF has been produced with Ghostscript versions older than 9.15.

  \item If the \LaTeX{} $\rightarrow$ \verb+dvips+ $\rightarrow$ \verb+ps2pdf+/Distiller route is being taken, make sure that the original graphics size (i.\,e. not scaled by any of the `{\tt scale}', `{\tt width}', `{\tt height}' or `{\tt depth}' options) does not exceed the page size of the final document. During PS to PDF conversion every graphic of the animation is temporarily moved to the upper left page corner. Those parts of the graphics that do not fit onto the document page will be clipped in the resulting PDF. Fortunately, graphics files for building animations may be resized easily to fit into a given bounding box by means of the `{\tt epsffit}' command line tool:

\quad{\tt epsffit -c <llx> <lly> <urx> <ury> infile.eps outfile.eps}

{\tt <llx> <lly> <urx> <ury>} are the bounding box coordinates of the target document. They can be determined using Ghostscript. For a document named `document.ps' the command line is

\quad{\tt gs -dNOPAUSE -q -dBATCH -sDEVICE=bbox document.ps}

Note that the name of the Ghostscript executable may vary between operating systems (e.\,g. `{\tt gswin32c.exe}' on Win/DOS).

  \item\hypertarget{dest:mem}{} Animations with complex graphics and/or many frames may cause \LaTeX{} to fail with a `\verb+TeX capacity exceeded+' error. The following steps should fix most of the memory related problems.

  MiK\TeX:
  \begin{enumerate}
    \item Open a DOS command prompt window (execute `cmd.exe' via `Start' $\rightarrow$ `Run').
    \item\label{item:firststep} At the DOS prompt, enter\\
    {\tt initexmf -{}-edit-config-file=latex}
    \item Type\\
    {\tt main\_memory=12000000}\\
    into the editor window that opens, save the file and quit the editor.
    \item\label{item:laststep} To rebuild the format, enter\\
    {\tt initexmf -{}-dump=latex}
    \item Repeat steps \ref{item:firststep}--\ref{item:laststep} with config files `{\tt pdflatex}' and `{\tt xelatex}'
  \end{enumerate}

  \TeX\ Live:
  \begin{enumerate}
    \item Find the configuration file `texmf.cnf' by means of\\
    {\tt kpsewhich texmf.cnf}\\
    at the shell prompt in a terminal.
    \item As Root, open the file in your favourite text editor, scroll to the `{\tt main\_memory}' entry and change it to the value given above; save and quit.
    \item Rebuild the formats by\\
    {\tt fmtutil-sys -{}-byfmt latex}\\
    {\tt fmtutil-sys -{}-byfmt pdflatex}\\
    {\tt fmtutil-sys -{}-byfmt xelatex}
  \end{enumerate}

  \item If you are postprocessing the created PDF file with tools such as pdftk to split the document into different parts, then the animation may fail. To work around this, do not use the `{\tt useocg}' (OCGs, PDF layers) option.  In addition, the control buttons also use OCGs to change their appearance to provide feedback about the running state, independent of the `{\tt useocg}' option. The workaround for this is not to use the `{\tt controls}' option.

  \item Animations should not be placed on \emph{multilayered} slides created with presentation making classes such as Beamer or Powerdot. Although possible (on the last overlay of a slide, at best), the result might be disappointing. Put animations on flat slides only. (Of course, slides without animations may still have overlays.)
\end{itemize}

\section{Acknowledgements}
I would like to thank Fran\c{c}ois Lafont who discovered quite a few bugs and made many suggestions that helped to improve the functionality of the package. Many thanks to Jin-Hwan Cho, the developer of `\verb+dvipdfmx+', for contributing the `\verb+dvipdfmx+' specific code, and to Walter Scott for proof-reading the documentation.

\begin{thebibliography}{8}
\bibitem{chupin} Chupin, M.: \emph{Syracuse MetaPost/Animations.} URL: \url{http://melusine.eu.org/syracuse/metapost/animations/chupin/?idsec=scara}
\bibitem{gilg05} Gilg, J.: PDF-Animationen. In: \emph{Die \TeX nische Kom\"odie}, Issue 4, 2005, pp.~30--37
\bibitem{hol} Hole\v{c}ek, J.: \emph{Animations in a pdf\TeX-generated PDF}. URL: \url{http://www.fi.muni.cz/~xholecek/tex/pdfanim.xhtml}
\bibitem{luque12} Luque, M.: \emph{PSTricks : applications.} URL: \url{http://pstricks.blogspot.com}
\bibitem{media9} \emph{The media9 Package}. URL: \url{http://www.ctan.org/pkg/media9}
\end{thebibliography}

\end{document}
